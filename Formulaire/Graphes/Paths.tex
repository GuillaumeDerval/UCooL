\subsection{Dijksta}
Plus court chemin d'un noeud $v$ à tout les autres. Le graphe doit être sans cycles de poids négatif.\\

\begin{lstlisting}
double[] dijkstra(LinkedList<Edge>[] g, int v)
{
  double[] d = new double[g.length];
  Arrays.fill(d, Double.POSITIVE_INFINITY);
  // initialize distance to v and the priority queue
  d[v] = 0;
  PriorityQueue<Edge> PQ = new PriorityQueue<Edge>();
  for(Edge e : g[v])
    PQ.add(e);
  while(!PQ.isEmpty())
  {
    // poll minimum edge from PQ
    Edge minE = PQ.poll();
    if(d[minE.dest] == Double.POSITIVE_INFINITY)
    {
      // set the distance to the new found endpoint
      d[minE.dest] = minE.w;
      for(Edge e : g[minE.dest])
      {
        // add to the queue all edges leaving the new
        // endpoint with the increased weight
        if(d[e.dest] == Double.POSITIVE_INFINITY)
          PQ.add(new Edge(e.orig, e.dest, e.w + d[e.orig]));
      }
    }
  }
  return d;
}
\end{lstlisting}
\subsection{Bellman-Ford\label{BellmanFord}}
Plus court chemin d'un noeud $v$ à tout les autres. Le graphe peut avoir des cycles de poids négatif, mais alors l'algorithme ne retourne pas les chemins les plus courts, mais retourne l'existence de tels cycles.\\
$d[i][u] = \text{shortest path from $v$ to $u$ with $\leq i$ edge}$\\
$d[0][v] = 0$ \\
$d[0][u] = \infty \quad \text{for $u \neq v$}$\\
$d[i][u] = \min \{ 
d[i - 1][u], \quad
\min_{(s, u) \in E} d[i - 1][s] + w(s, u) \}$\\
Si pas de cycle, la solution est dans $d[|V|-1]$. Si cycle il y a, $d[|V|-1]=d[V]$.\\
$O(|V||E|)$.\\

\begin{lstlisting}
double[] bellmanFord(LinkedList<Edge>[] gt, int v)
{
  int n = gt.length;
  double[][] d = new double[n][n];
  for(int u = 0; u < n; u++)
    d[0][u] = u == v ? 0 : Double.POSITIVE_INFINITY;
  for(int i = 1; i < n; i++)
  {
    for(int u = 0; u < n; u++)
    {
      double min = d[i - 1][u];
      for(Edge e : gt[u])
        min = Math.min(min, d[i - 1][e.dest] + e.w);
      d[i][u] = min;
    }
  }
  return d[n - 1];
}
\end{lstlisting}
\begin{lstlisting}
static boolean spfa (LinkedList<Edge>[] g, int s) {
  int n = g.length;
  int[] dist = new int[n];
  Arrays.fill(dist, Integer.MAX_VALUE);
  Queue<Integer> q = new LinkedList<Integer>();
  BitSet inQueue = new BitSet(n);
  int[] timesIn = new int[n];

  dist[s] = 0;
  q.add(s);
  inQueue.set(s);
  timesIn[s]++;
  while (!q.isEmpty()) {
    int cur = q.poll(); inQueue.clear(cur);
    for (Edge next : g[cur]) {
      int v = next.d, w = next.w;
      if (dist[cur] + w < dist[v]) {
        dist[v] = dist[cur] + w;
        if (!inQueue.get(v)) {
          q.add(v);
          inQueue.set(v);
          timesIn[v]++;
          if (timesIn[v] >= n) {
            return true;
          }
        }
      }
    }
  }
  return false;
}
\end{lstlisting}
\subsection{Floyd-Warshall}
Plus court chemin de tout les noeuds à tout les autres. Prend en argument la matrice d'adjacence. $O(|V|^3)$ en temps et $O(|V|^2)$ en mémoire.\\
Le graphe contient des cycles de poids négatif ssi $result[v][v]<0$.\\

\begin{lstlisting}
double[][] floydWarshall(double[][] A)
{
  int n = A.length;
  // initialization: base case
  double[][] d = new double[n][n];
  for(int v = 0; v < n; v++)
    for(int u = 0; u < n; u++)
      d[v][u] = A[v][u];

  for(int k = 0; k < n; k++)
    for(int v = 0; v < n; v++)
      for(int u = 0; u < n; u++)
        d[v][u] = Math.min(d[v][u], d[v][k] + d[k][u]);
  return d;
}
\end{lstlisting}
